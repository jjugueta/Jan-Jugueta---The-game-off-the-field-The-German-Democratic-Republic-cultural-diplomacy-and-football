
%%%%%%%%%%%%% Abstract

\chapter*{Abstract}

This thesis aims to make an intervention in cultural diplomatic studies by demonstrating sport’s relevance to the topic, thus contributing to the expansion of the scope and insights of this field of research. Despite sport’s significance within East German society, sport’s role within cultural diplomacy is under-researched. As an alternative to using coercive force, cultural diplomacy provides states with a soft-power approach to influence a foreign audience. Existing literature on cultural diplomacy focuses predominantly on how states during the 20th century channelled their cultural diplomatic efforts through the medium of the arts, music and dance. In the context of the Cold War, East German athletes’ performances in front of foreign audiences can be viewed as a form of cultural diplomacy. Rather than a general focus on all sports, the thesis examines cultural diplomatic efforts through football due to its importance to the GDR and its citizens. Labelled a sporting ‘miracle’, the GDR viewed their success in international football competitions as integral in increasing the state’s legitimacy and prestige. Mega-events such as the Olympics and FIFA World Cup were opportunities where the GDR could showcase the talent and ability of their athletes to the rest of the world. GDR athletes were not only representative of the nation’s best sporting talents, they also were presented as epitomising ‘ideal’ East German cultural characteristics. 

Using a Constructivist Grounded Theory method, articles reporting on football matches published in the GDR newspaper \textit{Neues Deutschland} are analysed for underlying social processes. The iterative nature of Constructivist Grounded Theory allows for the construction of a theory ‘grounded’ in the data. The resultant theory aids in explaining how \textit{Neues Deutschland} journalists ascribed desired qualities in GDR footballers, in turn examining how the state’s cultural diplomatic efforts were reflected in the newspaper.

