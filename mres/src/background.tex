\chapter{The German Democratic Republic, cultural diplomacy, media and football\label{cha:background}}

The purpose of the current chapter is to provide an overview of contextual information that is fundamental for understanding political factors that influenced the content of publications produced by state-controlled media. First, the history of the GDR, its position in the Cold War and system of governance and control will be explored, followed by a discussion of the media landscape, journalist education and media consumption. By doing so, the chapter will both frame the context of the state in the early 1970s – a time when the GDR received international recognition as a sovereign state - and provide an insight into the world that \textit{Neues Deutschland} operated within. The role and function of football in the GDR will then be discussed, highlighting the sport’s importance to the country in relation to cultural diplomatic goals and \textit{Leistungssport} (competitive sports). By combining how the GDR controlled their media landscape with how they portrayed the GDRNT, the chapter will demonstrate that the state was suitably positioned to present their cultural diplomatic efforts through football in \textit{Neues Deutschland}.

\section*{The German Democratic Republic}

The German Democratic Republic was a sovereign state that existed between 1949 and 1990. The People’s Council, with Moscow's approval, founded the GDR on October 7, 1949 (\cite{dennis2000}, p. 47). The GDR emerged from the Soviet-controlled zone after Germany was partitioned by the Allied powers following the defeat of the National Socialist regime at the end of World War II (\cite{dennis2000}, pp. 13-15). Prior to the formation of the GDR, the Soviet-controlled zone of Germany was administered by the \textit{Sowjetische Militäradministration in Deutschland} (Soviet Military Administration in Germany, or SMAD) (\cite{foitzik2009}). The GDR ceased to exist when it was integrated into an enlarged Federal Republic of Germany on October 3, 1990 (\cite{fullbrook1995}, p. 263). Soviet influence on the GDR was an ever-present force for the state’s entire existence.

The GDR found itself at the geographical frontline of the Cold War, located at the boundary which separated the Eastern and Western blocs on the European continent. The period of the Cold War between 1947 to 1991 witnessed a geopolitical struggle for global influence between the United States and the Soviet Union based on their ideological differences. At the crux of the Cold War was the conflict of ideas between the Western Blocs’ ideology of liberal capitalism and the Eastern Bloc’s brand of Marxist-Leninist communism (\cite{sargent2015}). The GDR’s allegiance was firmly rooted in the Eastern Bloc. As a founding member of the Warsaw Treaty Organisation (Warsaw Pact), the GDR was part of a military alliance with other European communist states that included the Soviet Union and stood in opposition to the American-backed North Atlantic Treaty Organisation (NATO) (\cite{wagner2012}).

The GDR held a unique status during the Cold War period as it was one half of a divided German nation. Whereas the GDR was aligned with the Eastern Bloc and a member of the Warsaw Pact, the FRG aligned itself with the Western Bloc and was a member of NATO (\cite{wagner2012}). The Basic Treaty was signed between the FRG and GDR in December 1972, formally recognising the GDR’s status as a sovereign nation in the eyes of the FRG (\cite{dennis2000}, p. 170). The internal division of the German nation into two sovereign and opposing states led to ongoing tensions between the GDR and FRG in the realms of politics, defence, culture and sport.\footnote{For more information regarding tensions between the FRG and GDR, Plock’s \textit{East German-West German Relations and the Fall of the GDR} (\citeyear{plock1993}) chronicles East German-West German relations drawing on a wide variety of academic and media publications. For further reading, Fullbrook’s \textit{Anatomy of a Dictatorship} (\citeyear{fullbrook1995}) and Dennis’ \textit{The Rise and Fall of the German Democratic Republic 1945-1990} (\citeyear{dennis2000}) provide comprehensive recounts of GDR history. Whilst FRG-GDR relations were not the focus of either books, Fullbrook and Dennis both provide fascinating insights into the relations between the two states. In addition, Carr’s article \textit{The Involvement of Politics in the Sporting Relationships of East and West Germany, 1945-1972} (\citeyear{carr1980}) offers an examination of the intersection of sport and politics between the FRG and GDR. Written and published during the existence of the GDR, it provides readers with an in-context understanding of the issue.}

Owing to the influence of the Soviet Union on the formation of the state, the GDR was a communist state. The GDR relied on a Marxist-Leninist ideology when governing the state (\cite{grixcooke2002}, p. 17). Labelled as a dictatorship by both Fullbrook (\citeyear{fullbrook1995}) and Dennis (\citeyear{dennis2000}), the GDR was governed by the SED in what was, in effect, a one-party system.\footnote{Although there was a plurality of political parties present in the GDR, elections between 1950 and 1990 were rigged in favour of the SED, which won a majority of seats in the People’s Chamber (\cite{dennis2000}, pp. 53-54).}  The economy was centrally controlled by the party apparatus (\cite{quint2012}, pp. 125-126). The SED had the defined goal of constructing state socialism in the GDR and saw itself as the ‘conscious and organised vanguard’ of the German labour movement and the ‘highest form of class organisation of the working class’ (\cite{dennis2000}, p. 54).

The SED, through their party apparatus and affiliated organisations, had developed structures of power that allowed them to control nearly all aspects of East German society. The \textit{nomenklatura} system imported from the Soviet Union enabled the SED to fill important positions in politics and industry with loyal cadres (\cite{dennis2000}, p. 197). This system facilitated the directives of the \textit{Zentralkomitee}\footnote{The Zentralkomitee was the body that was responsible for the management of the SED. Power was centred on the \textit{Sekretariat des Zentralkomitees} (Central Committee Secretariat) which concentrated on the details of policy and administration (\cite{dennis2000}, p. 191).} (Central Committee) and \textit{Politbüro}\footnote{The \textit{Politbüro} was an SED organisation that focused on the overall political strategy of the party. It dealt with important party issues that involved the economy, foreign policy, security, education, social policy, culture and cadre appointments (\cite{dennis2000}, p. 192).} to filter down to regional governmental authorities and mass organisation leadership (\cite{fullbrook1995}, p. 44). The \textit{Erster Sekretär des ZK} (First Secretary of the CC, later the \textit{Generalsekretär des ZK}, or General Secretary of the CC) chaired the \textit{Zentralkommitee} and held ultimate authority over the SED and therefore the state (\cite{dennis2000}, p. 191). With the appointment of cadres who were \textit{linientreu} (loyal to the party line), the SED solidified its position as the dominant force in East German politics and society.

The GDR developed a web of coercion and control over its citizens through its use of the military and the \textit{Staatssicherheitsdienst} (State Security Service, or Stasi). The Stasi had the important role of preventing ‘all attempts to delay or to hinder the victory of socialism’ (\cite{fullbrook1995}, p. 47). Intelligence gathering was the primary objective of the Stasi. They compiled approximately 6 million records on individuals during their existence (\cite{schroeder1998}, p. 444). The Stasi also relied on \textit{Inoffizielle Mitarbeiter} (unofficial collaborators/co-workers, or IM) to report on their fellow citizens if they were acting against the interests of the state. The number of IMs hit a peak of between 170,000 and 180,000 in the 1980s and were active in all aspects of East German society (\cite{fullbrook1995}, pp. 49-51; \cite{dennis2000}, pp. 212-215). The militarised border police were also a coercive force on GDR citizens. Their main role was to protect the GDR from external threats; however, they were authorised to use firearms on GDR citizens if they were otherwise unable to prevent the act of \textit{Republikflucht}\footnote{\textit{Republikflucht} was the act of escaping the GDR. Citizens who committed or attempted \textit{Republikflucht} were seen as traitors of the GDR. There were many differing motivations for a citizen wanting to commit \textit{Republikflucht}, from wanting to escape political oppression, seeking better economic opportunities in the West or wanting to be re-united with relatives in the West. For more information, see Major’s chapter ‘Crossing the Line: Republikflucht between Defection and Migration’ in his book \textit{Behind the Berlin Wall: East Germany and the Frontiers of Power} (\cite{major2009}).} (flight from the republic) (\cite{dennis2000}, p. 226). The number of coercive forces that were present in the GDR has led it to be labelled as a ‘police state’ (\cite{fullbrook1995}, p. 45).

\section*{Media in the GDR}

As with most institutions and sites of information production in the GDR, the state sought control over the media landscape. The role of the media in the GDR was to support the ruling party in the battle for the minds of its citizens (\cite{meyenfiedler2013}). Almost all East German media (except church-run newspapers) was controlled by the SED \textit{Politbüro} (\cite{vonharpe1997}, p. 186). The \textit{nomenklatura} system installed cadres loyal to the SED to leadership positions in media organisations (\cite{vonharpe1997}, p. 187). There was no official censorship apparatus, the term itself being considered taboo (\cite{jäger1993}, p. 18). Rather, the editors within media organisation were relied upon to perform unofficial acts of censorship (\cite{vonharpe1997}, p. 187). This system of unofficial censorship enabled the SED, through the work of loyal editors and functionaries, to steer the content of media publications and broadcasts so that it ‘guaranteed the socialist point of view would be represented’ (\cite{costabile-heming2000}, p. 54).

The education of journalism in the GDR was designed to produce journalists who could present the GDR and socialism in a positive light. Meyen and Fiedler liken journalists in the GDR to the PR department\footnote{Public Relations (PR) departments manage the spread of information between the organisation they represent and the public.} of a large company (\citeyear{meyenfiedler2013}, p. 331). This comparison is based in the belief that GDR journalists were ‘professional constructors of fictional realities’ whose objective was to ‘manipulat[e] the public’s perception to their purpose’ (Merten, cited in \cite{meyenfiedler2013}, p. 331). Journalists in the GDR were trained at the University of Leipzig, which attracted both students and professors who were social climbers interested in gaining access to the centres of power: the politicians (\cite{meyenfiedler2013}; \cite{meyenwiedemann2017}). Training at the University of Leipzig included instruction in working with language that embodied Marxist ideology (\cite{holzweißig2002}; \cite{kurzmüllerpötschkepöttker2010}). With their training, the GDR journalists did not view themselves as impartial writers, but as ‘advocates of both socialism and the GDR’ (\cite{meyenfiedler2013}, p. 332).

There was an abundance of print media present in the GDR, monopolised by the SED. The GDR had 39 daily newspapers, with the \textit{Neues Deutschland} and \textit{Junge Welt} having a combined circulation of more than one million readers (\cite{fiedlermeyen2015}, p. 452). However, the GDR faced persistent problems with shortages in paper resources, which meant the demand for newspapers could hardly be met (\cite{dussel2011}, p. 201). On average, each East German family read 1.5 newspapers, 1.4 weekly papers and more than 3 magazines (\cite{hanke1990}, p. 184). A wide range of interests was catered for with a variety of content in the many newspapers and magazines offered to East Germans (\cite{meyen2003}). As the mouthpiece for the SED, \textit{Neues Deutschland} communicated the opinions of the party to its audience, even containing the occasional editorial written by the \textit{Erster Sekretär des ZK} (\cite{fiedlermeyen2015}, p. 458). Despite the propaganda present in the newspaper, East Germans preferred \textit{Neues Deutschland} for their news as it was ‘rundum informativ’ (in every respect informative) (\cite{meyenschweiger2008}, p. 86). The audience for \textit{Neues Deutschland} extended beyond the boundaries of the state. Foreign statesmen and ambassadors read the content of \textit{Neues Deutschland} as official communications from SED party leadership (\cite{fiedler2014}, pp. 104-109).

With the proliferation of television sets in the GDR, East Germans were experiencing new ways to consume mass media. The proximity between the GDR and the FRG meant that East Germans were able to access West German television and radio broadcasts, expanding the amount of media they were able to consume (\cite{meyenschwer2007}). Unlike newspapers, GDR authorities could not control the transmission of foreign broadcasts in the GDR. The GDR had the lowest inhabitants per television set figures in the Eastern Bloc with 16.8 people per set in 1960, dropping to 3.2 in 1975 (\cite{lovell2017}). These circumstances meant that the battle between the GDR and FRG for East German minds played out in the living rooms of GDR citizens. In the safety of their homes, GDR citizens could choose which news programs they consumed from either state. Not all East Germans were interested in politics, with some preferring ‘soaps, movies and shows’ (\cite{meyenschwer2007}, p. 293). With exposure to contrasting perspectives from both East and West, not all East Germans ‘totally trusted the GDR media’ with many believing the truth to be existent ‘somewhere in between’ what East German and West German programs presented (\cite{meyenschwer2007}, p. 299).

\section*{The GDR and cultural diplomacy}

The GDR used cultural diplomacy to favourably present the state to foreign audiences. Cultural diplomacy enabled the GDR to communicate with foreign audiences where high-level diplomatic channels could not facilitate. With an absence of international recognition of sovereignty until the 1970s, the GDR sought ways to distinguish itself from its western German neighbour, which included cultural diplomatic programmes. The \textit{Gesellschaft für kulturelle Verbindugen mit dem Ausland} (Society for Cultural Relations with Foreign Countries) implemented and guided the GDR’s foreign cultural policy (\cite{hillaker2020}, p. 375). Print media was the primary medium for its cultural diplomatic efforts, producing two major magazines, \textit{DDR} and \textit{DDR Revue}. They were distributed to socialist and non-socialist states presenting the GDR as a peaceful and sovereign state (\cite{hillaker2020}).

Sport was another medium through which the GDR made cultural diplomatic approaches. Although the Olympic movement was designated to be apolitical, sports officials in the GDR leveraged sport for political means (\cite{balbier2009}). With full membership to the International Olympic Committee denied to the GDR until 1965, East German athletes had to compete in a joint all-German team during 1956-64 (\cite{balbier2009}). Once full membership was granted, the GDR saw the opportunity to reflect in sport the superiority of state-sponsored socialism (\cite{frenkin1964}). A large bureaucratic system was designed and implemented to identify and train athletes for the purpose to achieve success in international sport (\cite{dennisgrix2012}). To gain further on-field advantage, the GDR went to the extremes of doping their athletes (ibid). Success in international sport led to the raising of the GDR flag and playing of the anthem, which were symbols of East German identity, when athletes received their medals (\cite{balbier2009}).

\section*{Football in the GDR}

Football was a popular sport in the territorial area of the GDR both prior to and during the GDR’s existence. The sport’s history in the region dates back to the late 19th century when German Anglophiles imported it from the United Kingdom (\cite{mcdougall2014}, p. 14). Advocates of the game established football clubs that represented (and referenced in their names) the communities they emerged from. The sport became organised with the establishment of the \textit{Deutscher Fußball-Bund} (German Football Association) in the east German city of Leipzig in 1900 (\cite{peifferschulze-marmeling2008}, p. 17). Due to the sport’s rising popularity in the early 20th century, new stadiums with large attendance capacities were constructed in many east German cities (\cite{mcdougall2014}, p. 19). Football’s growth in the region was impacted by the Second World War through the drain of players called to the front and the destruction of football facilities (\cite{fuge2009}, p. 13).

Football in the GDR was influenced by the socialist goals of the state. The partition of Germany by the Allied Powers led to a restructuring of football in the region. In line with the shift towards a socialist society, the \textit{Deutscher Sportausschuß} (German Sports Committee) ordered traditional football clubs to break ties with the past and embrace new names that reflected the new socialist reality (\cite{mcdougall2014}, p. 39). Newer \textit{Betriebssportgemeinschaften}\footnote{The typical name for a BSG team would contain the prefix BSG followed by the industry which the company belonged and its locale. For example, BSG Lokomotiv Leipzig was the team from that represented the railroad workers from Leipzig. BSG Motor Zwickau represented the workers from the automotive industry in Zwickau. McDougall’s \textit{The People’s Game} (\citeyear{mcdougall2014}) contains more information regarding the BSG teams in the GDR.} (company sports clubs, or BSG) were introduced to the top-flight East German football competition. These BSGs would contain the name of their sponsoring company’s industry  along with its locale. The shift away from the traditional football club toward the BSGs signified the GDR’s desire to proliferate the idea of \textit{Arbeitersport} (workers’ sport) in the country.

Sport in the GDR maintained strict amateur status throughout its existence. Players who represented their BSGs did so in their ‘spare time’ and earned their living from working their official jobs (\cite{mcdougall2014}, pp. 67-68). Even with their amateur status, the standard wage for a footballer who played in the GDR’s top-flight competition in 1975 was 1,500 Marks per month, well over the 898 Mark per month wage of the average factory worker (\cite{wolle1998}, p. 49). Company leaders and functionaries wanting their team to do well occasionally made illegal payments to players as performance bonuses (\cite{mcdougall2014}, pp. 66-71). The reality presented here is that top-level GDR footballers were amateur only in name. The amateur status existent in the GDR helped to reinforce the idea that all elite sportspeople were still classified as fellow workers in the self-styled Workers’ and Peasants’ state (\cite{majorosmond2002}, p. 8).

Football was one of the many sports the GDR used to increase its prestige in the international arena. Historically, sport has been used as a political tool by states wishing to promote its national values (\cite{riordan1998}, p. 1). Through sport, the GDR aimed to demonstrate the supremacy of the socialist system over capitalism (\cite{dennis2000}, p. 208). To assist with this objective, only athletes who embodied ‘the best qualities of socialist personalities’ (\cite{mcdougall2014}, p. 61) could represent the GDR and become the nation’s ‘diplomats in tracksuits’ (\cite{dennisgrix2012}, p. 70). Only those who were \textit{linientreu} were allowed to travel abroad to play in international fixtures. As such, footballers’ political dependability became just as an important factor in their selection in the national team as their talent.

Football received considerable backing from the state, despite a lack of success in international competitions from 1949-1972. Football was classified as in the ‘Sport I’ funding category, which saw the GDR allocate more financial backing to the sport in line with the funding received by other sporting disciplines (where Olympic medal success was more likely) (\cite{dennisgrix2012}, p. 128). The funding allocation the national football team received highlights football’s importance to the state because the potential success the nation could achieve was disproportionate to the amount spent (unlike individual sports where the funding expenditure was much lower). The GDR also established a system of talent identification and training for young athletes that would ensure the nation would continue to produce elite sports stars (\cite{dennisgrix2012}, pp. 56-82).

\section*{Conclusion}

The information in this chapter illuminates the importance of football in the GDR, how it was leveraged by the state, and how it was represented in the media. Football in the GDR is exemplary of how the state used sport for its own political goals. Like most aspects in East German society, football was not only controlled by the state, it was exploited. The media representations of football in the GDR helped to present the narrative GDRNT footballers as existing on the elite end of the \textit{Arbeitersport} spectrum, and also as workers. The next chapter demonstrates how these representations are evident in the writing of sports journalists, found in match reports published in \textit{Neues Deutschland}.