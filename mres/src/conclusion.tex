\chapter{Conclusion\label{cha:conclusion}}

This thesis investigated how the GDR’s cultural diplomatic efforts through football were reflected in the state-run newspaper \textit{Neues Deutschland}. The findings demonstrate that \textit{Neues Deutschland} journalists followed processes of attributing desired qualities to GDR footballers when reflecting the state’s cultural diplomatic efforts through the sport. These desired qualities have an inherent association to a Prussian heritage the GDR sought to reclaim. The period from which the examined articles have been sourced suggests that a positive reconceptualisation of Prussia, knowingly or unknowingly, existed in the GDR some years before the \textit{Preußenwelle} of the late 1970s and early 1980s.

Employing a methodology used predominantly for analysing interviews in sociological studies could very well have been a risky endeavour due to the lack of studies demonstrating its successful application on extant texts. However, using Constructivist Ground Theory has enabled me to discover processes relating to ascription and evaluation embedded in the data, allowing for the construction of a theoretical model explaining how cultural diplomacy through football was reflected in \textit{Neues Deutschland}. Whilst this thesis is structured in an order that presents the research findings linearly, it is not representative of how the research process was conducted. Staying true to the methodology informed by Constructivist Grounded Theory, no hypotheses were (knowingly) formed prior to commencing the research process. If such hypotheses had to be offered, they would have made an obvious reference to socialism and thus influenced the research to lean that way. Fortunately, Charmaz’s prerequisite of having no ‘preconceived ideas and theories’ (\cite{charmaz2014}, p. 32) was followed, and this facilitated the discovery of how \textit{Neues Deutschland} journalists linked the GDR with a Prussian heritage. Chapter 3 has shown how this link was discovered. Chapters 4 and 5 reveal that \textit{Neues Deutschland} journalists were not motivated by the ideological struggle existent in the Cold War geopolitical context, but by the desire to present the GDR as having a distinct East German cultural identity derived from its Prussian heritage that differentiated them from the FRG.

This thesis shows that \textit{Neues Deutschland} journalists not only presented GDRNT footballers with desired qualities associated with a Prussian heritage, but that they were also able to be critical of these footballers when performances did not meet set expectations. The main actors in play were the sports journalists at \textit{Neues Deutschland}. Although they were writing for the SED’s own newspaper, the articles the journalists published should be seen as their own work. They may have inevitably produced content that aligned with the ideology of the state, but they appear to have done so under an unofficial system of self-censorship. Interestingly, unlike the topics of politics, the economy or foreign relations, sport allowed journalists to perform an act one would think to be taboo in a totalitarian state: criticise sportspeople who represented the GDR. Sport could (sometimes) transcend the bondages of media control in the GDR since sports journalists were still inherently football fans and acted as such. Being read as the mouthpiece of the SED, foreign statesmen and audiences could have incorrectly interpreted the criticisms found in \textit{Neues Deutschland} match reports as coming directly from party leadership.

This thesis demonstrates that sport is a research topic in which cultural diplomacy can be studied. Furthermore, this thesis provides a foundation for further research into how the GDR’s cultural diplomatic efforts were reflected in their printed media and deployed through non-traditional sites of cultural production such as football. \textit{Neues Deutschland’s} function changed depending on the position of the reader. It served a cultural diplomatic purpose for a foreign audience. For the domestic audience, it was another form of control from the state. It proliferated the state-endorsed socialist ideology to the citizenry as propaganda, presenting to them what the East German athlete should be. The scope of the research only allowed for a limited time period to be analysed. Prospective researchers could focus on earlier publications of \textit{Neues Deutschland} to discover whether similar associations to a Prussian heritage existed in those times. Match reports published after the period examined could also be analysed to ascertain whether there was a change in what was being ascribed to GDR footballers after their historic win against the FRG. Alternatively, researchers could undertake a similar study to this thesis using the GDR’s other major newspaper \textit{Junge Welt}, which catered for a younger audience, was allegedly significantly more journalistic than \textit{Neues Deutschland} and made greater efforts to respond to the wishes of the public (\cite{meyenschweiger2008}).

Sport (and football especially) continues to provide academia with a topic rich with research possibilities. It has a long history and is played by humans in every corner of the globe. Whilst its form and rules may change with respect to its different disciplines, what remains the same is our fascination with it. It is an important part of our societies and cultures. As such, cultural studies of sport should continue to be undertaken, for they may allow for further discoveries that demonstrate the wonderment of humans at play.